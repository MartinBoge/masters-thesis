\thispagestyle{plain}
\section{Introduction}

Rising carbon emissions are destabilizing our climate system, causing global temperature increases, extreme weather events, and threatening ecosystems worldwide. A rapid and comprehensive transition to green energy sources is therefore essential to prevent further environmental degradation and secure a sustainable future for coming generations. To effectively manage the transition to green energy systems an element is the ability to accurately forecast short-term carbon emissions. These forecasts function as a practical tool in production planning for manufacturing companies and in grid management for a Transmission System Operator (TSO).

Manufacturing companies, particularly in energy-intensive sectors, are increasingly focusing on reducing their carbon footprints and complying with stringent emission regulations. By aligning energy-intensive processes with periods of lower carbon emissions, manufacturers can reduce their carbon footprint. For example, AI-driven scheduling can shift high-energy tasks to times when the electricity grid is greener, thereby enhancing sustainability \parencite{futurebridge}.

In the energy sector, a TSO is an organization responsible for operating, maintaining, and developing the grid supplying electricity and natural gas in a particular region or country. The \citeauthor{entsoe2022} (ENTSO-E) emphasizes the necessity for enhanced system flexibility and advanced operational strategies to manage the increasing integration of renewable energy sources. In their vision for a carbon-neutral Europe, ENTSO-E highlights that the future power system will be highly weather-dependent, necessitating significant flexibility to maintain system adequacy and resilience. This underscores the critical role of accurate short-term carbon emissions forecasts in enabling TSOs to balance supply and demand effectively \parencite{entsoe2022}.

Considering this challenge, especially in weather-dependent power grids, an intriguing case is Denmark's western bidding zone (DK1). In DK1 wind power generation accounts for over 50 percent of its electricity mix \parencite{wang2017,iea2023}, one of the highest shares worldwide. Moreover, many power systems are setting similar ambitious targets with for instance Germany that aims to reach 80 percent renewable electricity by 2030 and 100 percent by 2035, with wind power playing a major role \parencite{iea2025}. The unique characteristics of DK1 make it an ideal setting for studying the relationship between renewable energy variability and carbon emissions. Unlike more hydropower-dominated grids, where storage capacity can smooth out fluctuations, DK1 further relies on market-based mechanisms, cross-border electricity trading, and flexible generation to maintain system stability \parencite{energistyrelsen}.

\subsection{Research Context}

The operational characteristics of DK1 with a high share of renewable energy sources create significant forecasting challenges that extend beyond traditional time series modeling capabilities. The volatile wind generation patterns, coupled with market-based balancing mechanisms and cross-border electricity flows, result in carbon emission dynamics characterized by nonlinear relationships, regime changes, and complex temporal dependencies that span multiple time scales \parencite{carlini2023}. Traditional statistical approaches such as autoregressive models, while valuable for their interpretability, are fundamentally limited in their ability to capture these intricate patterns and sudden shifts in generation mix composition \parencite{box2015, hyndman2021}.

Machine learning approaches designed for sequential data offer promising alternatives for addressing these modeling challenges. Long short-term memory neural networks (LSTM) have demonstrated particular effectiveness in energy forecasting applications due to their ability to selectively retain relevant information across different time horizons while adapting to new conditions \parencite{hochreiter1997}. The gating mechanisms of LSTMs enable it to maintain context about weather patterns, generation schedules, and market dynamics for appropriate durations, making it well-suited for the temporal complexity inherent in wind-dominated systems \parencite{ostermann2024,pierre2023}.

However, the existing literature reveals a significant research gap regarding LSTM applications in highly renewable-integrated bidding zones. Most carbon emission forecasting studies focus on larger national grids or systems with different renewable profiles, while the specific challenges of wind-dominated zones like DK1 remain largely unexplored \parencite{ostermann2024, bokde2021}. This gap is particularly significant given that DK1's operational characteristics are increasingly representative of future power systems as countries pursue aggressive renewable energy targets. Understanding how advanced neural network architectures perform in this challenging environment provides crucial insights for the growing number of power systems following similar renewable integration pathways.

\subsection{Problem Statement}

\newcommand{\researchquestion}{
    \begin{quote}
        \textit{How can long short-term memory neural networks be designed for short-term (1--24 hour) \cotwo{} emission forecasting in Denmark's wind-dominated DK1 bidding zone to improve prediction accuracy compared to simple benchmark models and traditional autoregressive models?}
    \end{quote}
}

Given the identified research gap and the critical operational needs of renewable-heavy power systems, this thesis addresses a fundamental challenge in modern grid management. The intersection of high wind penetration, volatile generation patterns, and complex market dynamics in DK1 creates carbon emission forecasting challenges that existing methodologies struggle to address effectively. While traditional statistical approaches remain valuable for their interpretability, their fundamental limitations in capturing nonlinear relationships and regime changes necessitate exploration of more sophisticated modeling approaches capable of handling the temporal complexity inherent in wind-dominated systems.

The main research question for this thesis becomes: \researchquestion{}

This question encompasses several critical dimensions that define the scope and approach of this thesis. The focus on "\textit{long short-term memory neural networks}" reflects the need for architectures capable of learning complex temporal dependencies and adapting to the dynamic operational conditions characteristic of renewable-heavy grids. The phrase "\textit{designed for}" emphasizes that this is not merely an application of existing LSTM models, but rather a systematic exploration of architectural configurations, hyperparameters, and design choices optimized specifically for carbon emission patterns in volatile wind systems. The "\textit{short-term (1--24 hour)}" specification addresses practical operational requirements, spanning from immediate decision-making support to day-ahead planning horizons essential for grid management and industrial scheduling. The geographic focus on "\textit{Denmark's wind-dominated DK1 bidding zone}" provides a strategically valuable case study whose operational characteristics are increasingly representative of future power systems worldwide. Finally, the commitment to "\textit{improve prediction accuracy compared to simple benchmark models and traditional autoregressive models}" establishes a rigorous evaluation framework that demands demonstrable performance gains over both naive approaches and established statistical methods.

The sub questions to help guide this thesis becomes:

\begin{enumerate}
    \item What architectural configurations and hyperparameters optimize LSTM performance for capturing carbon emission patterns in the DK1 zone?
    \item Which input features and feature engineering strategies most significantly enhance LSTM predictive performance for short-term carbon emissions forecasting?
    \item How does the forecasting horizon (1-hour vs 24-hour ahead) affect model performance and required architectural adaptations?
    \item What quantifiable improvements in prediction accuracy can LSTM models achieve compared to benchmark models and traditional autoregressive approaches for the DK1 zone?
    \item To what extent could manufacturing companies operating in DK1 and Energinet realize economic and carbon reduction benefits from enhanced carbon emission forecasting accuracy?
\end{enumerate}

These sub questions provide a systematic framework for addressing the main research question through technical, empirical, and practical dimensions. The first three sub questions guide the technical development: exploring LSTM design choices specific to carbon emission dynamics, directing attention toward optimal feature engineering strategies, and enabling systematic comparison across different forecasting horizons to reveal how model requirements evolve with prediction complexity. The fourth sub question establishes the empirical validation framework necessary to demonstrate quantifiable improvements relative to existing forecasting methods. The fifth sub question extends beyond technical performance to evaluate the practical economic and environmental benefits for key stakeholders in the DK1 system. Together, these sub questions structure the thesis to provide theoretical insights into LSTM capabilities, empirical validation of performance gains, and practical guidance for implementing advanced forecasting systems that deliver measurable value in renewable-heavy power grids.

\subsection{Thesis Structure}

This thesis is structured to systematically address the research question through a logical progression from theoretical foundations to empirical findings. Following this introduction, the literature review examines existing approaches to short-term carbon emission forecasting and energy prediction, identifying methodological insights and establishing the research gap that motivates this DK1-focused investigation. The theory section provides the mathematical and architectural foundations necessary for understanding neural networks and LSTM architectures, progressing from basic feed-forward networks through recurrent architectures to the specialized gating mechanisms that enable LSTMs to capture long-term temporal dependencies. The methodology section details the systematic approach employed for model development, including data collection and preprocessing procedures, LSTM architectural design choices, baseline model establishment, and evaluation frameworks for both a 1-hour and 24-hour forecasting horizon. The results section presents comprehensive empirical findings, encompassing baseline model performance, LSTM hyperparameter optimization outcomes, forecasting accuracy comparisons, and feature importance analysis that validates the theoretical foundations underlying the variable selection. The discussion synthesizes these findings to address the research question and sub questions, evaluating the practical implications for renewable energy forecasting while identifying limitations and directions for future research. Finally, the conclusion provides a concise summary of key findings, contributions to the field, and recommendations for future work in carbon emission forecasting for renewable energy systems.
