\thispagestyle{plain}
\section*{Abstract}
\addcontentsline{toc}{section}{Abstract}

Rising carbon emissions and the global transition to renewable energy have created an urgent need for accurate short-term carbon emission forecasting in volatile wind power grids. Denmark's western bidding zone (DK1), with over 50\% wind generation, exemplifies the forecasting challenges faced by modern renewable-dominated power systems. The volatile wind generation patterns, coupled with market-based balancing mechanisms and cross-border electricity flows, result in carbon emission dynamics characterized by nonlinear relationships and complex temporal dependencies that traditional statistical approaches struggle to capture effectively. This thesis addresses the fundamental question of how long short-term memory neural networks can be designed for short-term (1-24 hour) carbon emission forecasting in DK1 to improve prediction accuracy compared to simple benchmark models and traditional autoregressive approaches.

This thesis develops and evaluates two LSTM architectures using three years of hourly data (2022-2025) encompassing carbon emissions and explanatory variables including renewable generation forecasts, consumption patterns, market prices, and meteorological conditions. The methodology employs a single-layer LSTM for 1-hour ahead forecasting and an encoder-decoder architecture for 24-hour ahead prediction, with systematic hyperparameter optimization conducted through coordinate descent. Performance is evaluated against naive persistence and optimally-configured ARIMA baselines using statistical significance testing via Diebold-Mariano tests.

The results demonstrate substantial forecasting improvements across both prediction horizons. For 24-hour ahead prediction, the encoder-decoder LSTM achieves an 11.3\% reduction in RMSE compared to ARIMA models and a 25.4\% reduction compared to naive persistence, with all improvements confirmed as statistically significant. Feature importance analysis reveals that forecasted solar photovoltaic production emerges as the most influential exogenous predictor despite the dominance of wind-power in DK1. This interestingly reveals that renewable power sources with more predictable generation patterns provide a better signal compared to those of larger share of the total renewable generation mix.

The practical implications extend beyond academic interest to tangible economic and environmental benefits. Conservative estimates indicate that a medium-sized manufacturing company could achieve annual carbon reductions of 855 to 1,905 tonnes \cotwoe{} through optimized production scheduling enabled by accurate emission forecasts, representing cost savings of \euro72,700 to \euro161,900 at current carbon prices. For transmission system operators, enhanced forecasting enables proactive grid management strategies that could reduce emergency reserve activations, yielding operational savings of \euro3.2 to \euro7.1 million annually while avoiding 8,500 to 19,000 tonnes \cotwoe{} in emissions.

This thesis establishes LSTM neural networks as an effective solution for carbon emission forecasting in wind-dominated grids, addressing a significant research gap in renewable energy forecasting. The demonstrated accuracy improvements and quantified economic benefits provide compelling justification for deploying advanced forecasting architectures in industrial demand response systems and grid operations. As electricity systems worldwide pursue aggressive renewable integration targets similar to DK1's profile, the forecasting methodologies developed offer a deeper understanding of carbon emission forecasting in wind-dominated grids while supporting both economic efficiency and climate objectives in the transition to sustainable power generation.
