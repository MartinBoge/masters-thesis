\thispagestyle{plain}
\section{Discussion}

This discussion examines the implications of the LSTM model's superior forecasting performance for 24-hour carbon emission prediction in DK1. The analysis begins by exploring merit-order dynamics in wind-dominated grids, revealing counterintuitive insights about generation capacity versus forecasting informativeness. The discussion then quantifies practical benefits for manufacturing companies and Energinet, demonstrating substantial economic and environmental value. Finally, the findings are positioned within existing literature before examining methodological limitations and identifying promising future research directions.

\subsection{Merit-Order Dynamics and Carbon Emission Forecasting}

An interesting finding emerges from the feature importance analysis: despite DK1 being a wind-dominated grid with over 50\% wind generation \parencite{wang2017,iea2023}, future solar photovoltaic production ranks as the most influential exogenous variable for carbon emission forecasting (importance score = 4.94). This result reveals important insights about merit-order dynamics in volatile renewable systems. While wind generation provides the bulk of renewable electricity in DK1, solar production offers superior predictive value for carbon emissions due to its highly deterministic daily patterns. Solar generation follows predictable cycles with zero overnight production and peak midday output, providing the LSTM model with reliable temporal anchors that help structure expectations about daily emission patterns. In contrast, wind generation, despite its dominance in capacity terms, exhibits greater stochastic variability that may offer less systematic information for emission forecasting even though it drives larger absolute changes in carbon output. In other words, it is easier to predict when the sun is shining rather than when the wind is blowing.

This finding shows a fundamental distinction between generation capacity and forecasting informativeness within the merit-order framework. Solar forecasts appear to serve as an effective proxy for broader renewable availability and daily demand-supply dynamics, helping the model anticipate when the system will rely more heavily on carbon-intensive thermal backup generation. Solar's predictable daily cycle helps the model anticipate dispatch patterns. Zero solar generation overnight means greater reliance on thermal backup, while peak solar generation at midday reduces the need for carbon-intensive plants. This suggests that in wind-dominated grids, the most valuable forecasting variables for carbon emissions may not necessarily be those representing the largest generation sources, but rather those providing the most systematic and predictable signals about overall renewable availability and merit-order dispatch patterns.

\subsection{Practical Impact for Manufacturing Companies}
\label{subsec:practical-impact-for-manufacturing-companies}

Manufacturing companies, particularly in energy-intensive sectors such as steel, aluminum, and chemical production, increasingly utilize demand response strategies to align their production schedules with periods of lower carbon emissions \parencite{futurebridge}. Europe-wide modeling suggests that heavy-industry processes can flex at least 25\% of their electricity demand without affecting output \parencite{gils2014}. Assuming a medium-sized manufacturing company with annual electricity consumption of 50 GWh operating in the DK1 grid, the carbon footprint of the company varies significantly based on timing of energy consumption, with emission factors ranging from near-zero during high renewable periods to over 800 kg \cotwoe{}/MWh during thermal backup periods \parencite{energidataservice2025}.

To quantify the practical benefits of improved forecasting accuracy, consider that the 24-hour LSTM model achieves an 11.3\% reduction in RMSE compared to ARIMA and a 25.4\% reduction compared to naive persistence (see \autoref{sec:results}). Assuming that half of this forecasting improvement translates into optimized production scheduling (conservative estimate accounting for operational constraints), a manufacturing company could reduce forecast-related scheduling errors by approximately 5.7\% relative to ARIMA and 12.7\% relative to naive approaches. For the 50 GWh company with an assumed average emission factor of 300 kg \cotwoe{}/MWh in DK1, this improved scheduling could yield annual carbon reductions of 855 to 1,905 tonnes \cotwoe{}. At current EU ETS carbon prices of approximately \euro85 per tonne \cotwoe{} \parencite{abnett2022,europeancommission2023}, these emission reductions represent annual cost savings of \euro72,700 to \euro161,900 for a single medium-sized company.

Aggregating these benefits to the entire DK1 grid reveals substantial potential for carbon reduction through improved emission forecasting. The DK1 bidding zone consumes approximately 20 TWh of electricity annually, with industrial and manufacturing sectors accounting for roughly 37.5\% of total consumption (both figures derived via \textcite{energidataservice2025}). Assuming that 25\% of industrial consumption (approximately 1.875 TWh annually) could benefit from demand response scheduling enabled by accurate carbon forecasting, and applying the conservative 5.7\% improvement factor, the aggregate carbon reduction potential reaches 32,063 to 71,438 tonnes \cotwoe{} annually across the DK1 grid. At current carbon prices, this represents \euro2.7 to \euro6.1 million in annual economic value. A typical passenger vehicle emits 4.6 tonnes \cotwo{} in a year \parencite{usepa2023}, meaning emission reductions are equivalent to removing 6,970 to 15,530 passenger vehicles from Danish roads for one year.

Based on the stated assumptions and additional conservative assumptions about operational flexibility and demand response capabilities detailed in \autorefapdx{apdx:manufacturing-impact-analysis} (which also contains a detailed overview of the calculations), these results demonstrate that accurate carbon emission forecasting delivers tangible environmental and economic benefits that extend well beyond individual facilities to meaningful grid-wide impact. The monetary savings and carbon reductions provide compelling justification for advanced forecasting systems deployment across industrial sectors. As carbon pricing mechanisms expand and manufacturing companies face increasing regulatory pressure, sophisticated emission forecasting architectures like the LSTM model developed in this thesis represent essential infrastructure for achieving both economic efficiency and climate objectives in renewable-dominated power systems.

\subsection{Practical Impact for the Energinet}
\label{subsec:practical-impact-for-energinet}

Energinet, Denmark's transmission system operator, must maintain grid stability in a wind-dominated system where generation deviations from day-ahead schedules require activating manual frequency-restoration reserves (mFRR) using fossil-fuel units that emit approximately 500 kg \cotwoe{}/MWh \parencite{energidataservice2025}. In DK1, annual up-regulation energy averages 0.6 TWh, with activation prices sometimes exceeding 35,000 DKK/MWh during scarcity events \parencite{energinet2023}.
Accurate 24-hour carbon emission forecasts enable three operational strategies that reduce mFRR dependency. When high carbon emissions are predicted, indicating heavy thermal generation reliance, Energinet can proactively coordinate demand response with industrial consumers, schedule imports from cleaner neighboring grids, and optimize reserve positioning to reduce emergency activations.

The 24-hour LSTM model's superior forecasting accuracy, achieving an 11.3\% reduction in RMSE compared to ARIMA and a 25.4\% reduction compared to naive persistence (see \autoref{sec:results}), translates into better anticipation of grid conditions that typically trigger mFRR needs. Assuming that 25\% of this forecasting improvement enables operational optimization (accounting for the constrained nature of real-time grid operations and regulatory requirements), Energinet could reduce mFRR activations by approximately 2.8\% relative to ARIMA-based planning and 6.4\% relative to naive approaches. Applied to the 0.6 TWh annual baseline, this represents avoided mFRR energy of 17 GWh and 38 GWh respectively, corresponding to annual emission reductions of 8,500 to 19,000 tonnes \cotwoe{}.

At an average mFRR activation cost of \euro186 per MWh \parencite{energinet2024}, these avoided activations yield annual cost savings of \euro3.2 to \euro7.1 million for Energinet. While these emission reductions represent a modest fraction of Denmark's total power sector footprint, they equal removing 1,850 to 4,100 passenger vehicles from Danish roads annually, based on typical vehicle emissions of 4.6 tonnes \cotwoe{} per year \parencite{usepa2023}. Beyond direct cost savings, enhanced carbon emission forecasting enables Energinet to transition from reactive to proactive grid management, reducing system stress and improving overall grid reliability. Based on conservative assumptions regarding operational constraints detailed in \autorefapdx{apdx:energinet-impact-analysis}, these results demonstrate that accurate short-term carbon emission forecasting delivers tangible operational and economic benefits for transmission system operators, providing compelling justification for integrating advanced forecasting architectures like the LSTM model developed in this thesis into TSO operational procedures.

\subsection{Comparison with Existing Literature}

The results of this thesis align with and extend findings from the somewhat limited body of literature on short-term carbon emission forecasting. The 24-hour LSTM model's achievement of 22.20 tonnes \cotwoe{} RMSE compares favorably with regional-scale studies such as \textcite{leerbeck2020}, who reported normalized RMSE values of 0.095 to 0.183 for DK2's carbon emission forecasting using hybrid ARIMAX approaches. While direct numerical comparison is challenging due to different normalization methods and regional characteristics, both studies demonstrate that sophisticated modeling approaches can achieve meaningful accuracy improvements over naive baselines for Danish bidding zones. The LSTM's 11.3\% RMSE improvement over ARIMA is more modest than the dramatic performance gains reported by \textcite{ostermann2024} for German grid forecasting, where gradient boosting substantially outperformed SARIMAX models, yet remains statistically significant and operationally meaningful.

This thesis directly addresses the identified research gap by providing the first dedicated LSTM-based carbon emission forecasting study for a wind-dominated regional bidding zone. Unlike existing studies that focus on larger national grids \parencite{ostermann2024,bokde2021} or employ primarily statistical approaches \parencite{leerbeck2020}, this thesis demonstrates that neural network architectures can effectively capture the nonlinear dynamics specific to volatile wind power systems. The successful integration of day-ahead exogenous variable forecasts within the encoder-decoder architecture represents a methodological advance over previous regional carbon forecasting studies, while the feature importance analysis validates theoretical merit-order principles empirically, providing insights specific to wind-heavy systems that were absent from previous literature.

\subsection{Limitations and Critical Reflection}

Several methodological limitations constrain the scope and generalizability of this thesis's findings. The exclusive focus on LSTM architectures prevents comparative evaluation against other promising sequential models such as Transformer encoders, temporal convolutional networks, or gated recurrent units that may be equally or more suitable for carbon emission forecasting. Despite substantial literature evidence demonstrating superior performance of hybrid statistical-machine learning approaches, this thesis deliberately excludes ARIMA-LSTM combinations that consistently outperform standalone models in energy forecasting applications. The coordinate descent hyperparameter optimization, while systematic, represents a relatively simple search strategy that may miss optimal parameter combinations discoverable through more sophisticated approaches such as Bayesian optimization or genetic algorithms. The evaluation framework relies heavily on RMSE and MAE metrics without incorporating directional accuracy, prediction interval coverage, or performance under extreme operational conditions, which are metrics that may be equally critical for practical grid management applications.

The dataset characteristics and temporal structure impose additional constraints on the validity and generalizability of findings. The three-year observation period (2022-2025), while providing high-resolution hourly data, represents a relatively limited timeframe for training deep learning models and may not capture sufficient variability in long-term weather patterns, policy changes, or infrastructure developments that influence carbon emission dynamics. The reliance on Energy Quantified's proprietary synthetic data generation to fill missing observations introduces potential unknown biases that could systematically alter natural emission patterns in ways that influence both model learning and performance evaluation. The 70/15/15 temporal split allocates the most recent operational period to testing, potentially limiting the assessment of seasonal variations and model robustness across different operational regimes present in the full dataset. Finally, the DK1-specific focus, while addressing an identified research gap, constrains applicability to other renewable-dominated grids with different wind penetration levels, market structures, interconnection capacities, or merit-order dynamics, requiring careful consideration when extrapolating these findings to alternative regional contexts.

\subsection{Future Research Directions}

Several promising extensions could build upon these findings. Comparative studies incorporating Transformer architectures, temporal convolutional networks, and hybrid ARIMA-LSTM models would reveal whether alternatives can further improve forecasting accuracy in wind-dominated systems. Cross-regional validation across neighboring bidding zones such as DK2, southern Norway, or northern Germany could test the generalizability of LSTM approaches beyond DK1's specific operational characteristics. Probabilistic forecasting extensions that quantify prediction uncertainty through ensemble methods or confidence intervals would provide valuable operational information for risk-averse grid management decisions. Additionally, real-time deployment studies integrating these carbon emission forecasts into actual demand response platforms or TSO operational systems could validate the practical benefits estimated in this analysis while revealing implementation challenges. Finally, longer-term datasets spanning multiple years and diverse weather regimes could explore model robustness across seasonal variations and changing grid infrastructure, addressing limitations from the three-year observation period.
