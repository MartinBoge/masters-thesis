\thispagestyle{plain}
\section{Literature Review}
\label{sec:lit}

This section examines the academic literature on short-term (hourly and up to 24-hour ahead) carbon emissions forecasting, with a focus on the applied methodologies. Furthermore, it covers relevant discussions and findings related to the research question and related work within short-term energy forecasting (e.g. load and wind forecasting), since the techniques often carry over to predicting carbon emissions. Key questions addressed include: (1) What are the typical methodologies used for short-term carbon emissions forecasting or similar energy forecasts? (2) What methodological and practical insights from past studies are relevant for this thesis? (3) Why is DK1 an interesting and unique case compared to other regions? and (4) What research gap can be identified for a DK1-focused study? This literature review establishes the methodological foundation for this thesis while highlighting the research gap in carbon emissions forecasting for the DK1 region that this work aims to fill.

\subsection{Modeling Approaches in Short-Term Emissions and Energy Forecasting}

The methodologies in short-term forecasting methods span a spectrum from classical statistical techniques like autoregressive integrated moving average (ARIMA) \parencite{box2015}, exponential smoothing (e.g. Holt-Winters) \parencite{hyndman2021}, state-space/Kalman filter models \parencite{durbin2001} and structural decomposition \parencite{harvey1993} to more flexible machine-learning approaches such as linear regression \parencite{douglas2021} and nonlinear regression \parencite{seber2003}, random forests \parencite{breiman2001}, gradient-boosted trees (e.g. XGBoost) \parencite{chen2016} and support-vector regression \parencite{drucker1997}, and finally to deep-learning architectures tailored for sequential data, including recurrent neural networks (RNNs) \parencite{elman1990}, long short-term memory neural networks (LSTMs) \parencite{hochreiter1997} and gated recurrent unit models (GRUs) \parencite{cho2014}, temporal convolutional networks (TCNs) \parencite{bai2018} and Transformer-based encoders/decoders \parencite{vaswani2017}. Hybrid and ensemble schemes that blend statistical and learning-based models have also proven effective \parencite{zhang2003}.

In the literature there is a handful of studies that directly forecast carbon emissions or intensity. Expanding the scope to energy related forecasts there are even more to draw meaningful insights from. A closely related study to the forecasting problem in this thesis is \citetitle{ostermann2024} by \textcite{ostermann2024}, which forecasts the hourly carbon intensity of Germany's electricity mix up to 24 hours ahead. It applies a SARIMAX (seasonal ARIMA with exogenous variables) model as a representative traditional approach, and compares it with various machine learning models (bagging, random forests, gradient boosting) and deep learning models (CNN, LSTM, MLP) using 2019-2022 data. All the advanced models outperformed baseline persistence forecasts, for example, gradient boosting achieved the lowest mean absolute percentage error. The SARIMAX model performed respectably but was less accurate than the best non-parametric (ML) models. Notably, the authors observe that their deep learning models did not fully outperform simpler methods, cautioning that the added complexity of deep learning must be justified by significant accuracy gains. They suggest weighing implementation effort against incremental benefit when considering complex models in practice.

Another great example is the study \citetitle{bokde2021} by \textcite{bokde2021}. This study proposes a time-series decomposition method to forecast electricity-related \cotwo{} emissions on a short-term basis (up to 48 hours, aligning with the day-ahead market). The approach breaks the emissions time series into components (e.g. trend, seasonal, residual) and forecasts each component with appropriate models (statistical or machine learning), then recombines them. In performance benchmarks on national data, the proposed method delivered significantly improved accuracy - for instance, in France it achieved a MAPE about 25\% lower than the best existing state-of-the-art model. The authors demonstrate the method across five European countries (France, Germany, Norway, Denmark, Poland) and show that smart scheduling using these forecasts can substantially cut emissions (e.g. the 25\% reduction in France by shifting a 4-hour load block to greener periods). While the focus was on enhancing classical forecasting via decomposition, it implicitly compares against other models (some of which include machine learning), showing the decomposition strategy can rival or beat more complex approaches. However, they do not explore the use of deep learning.

Given this thesis focuses on DK1, the study by \textcite{leerbeck2020} is particularly relevant. This study forecasts hourly average and marginal \cotwo{} emission factors for the DK2 region (Eastern Denmark) with a 1--24 hour horizon. The methodology begins with a large set of 473 candidate features (including grid production, demand, weather from Denmark and neighbors) which is then pruned to \textless30 via LASSO and forward selection. The forecasting model itself combines three specialized linear regression models (to capture different nonlinear effects) into an ensemble, and then applies an ARIMA model to the residuals of the ensemble - effectively yielding an ARIMAX (ARIMA with exogenous inputs) , i.e., they construct a hybrid model. This traditional-meets-ML approach proved useful. The normalized RMSE ranged from 0.095 to 0.183 for average emission intensity and 0.029 to 0.160 for marginal intensity (1-hour ahead being most accurate). The model also produced well-calibrated prediction intervals, addressing the uncertainty in \cotwo{} forecasts. The authors note that marginal emissions in DK2 appeared uncorrelated with local features (suggesting marginal generators are often outside the zone), an insight gained from the linear model coefficients. They do not explicitly compare against deep learning, but the study demonstrates that a carefully constructed linear/ARIMA hybrid (with feature engineering) can achieve high accuracy. They mention that their hybrid model is generalizable to any European bidding zone with minimal manual tuning, implying that more complex models (e.g. neural networks) might not be necessary if a robust statistical framework is in place.

When expanding the scope and consider cases that are not necessarily forecasting carbon emissions, but more forecasting within the energy field there are a couple of articles worth mentioning. \textcite{pierre2023} has conducted a study, where they mix classical ARIMA with an LSTM. In the paper \citetitle{pierre2023} they present a model for short-term load forecasting - that is the prediction of electricity demand (load) over a short time horizon. More specifically, the authors predict the daily peak-hour electricity demand for the Beninese grid by modeling: a standalone ARIMA model, standalone LSTM and GRU neural networks, and hybrid models that combine ARIMA with LSTM or GRU. The results highlight the limitations of ARIMA in capturing volatile peaks - the pure ARIMA model had an RMSE of 49.90 (in relative units), whereas the LSTM and GRU alone brought the error down to 18.7 and 18.1 respectively. Most notably, the ARIMA-LSTM hybrid model achieved the best accuracy with an RMSE of 7.35, dramatically outperforming the single-model approaches. The hybrid works by using ARIMA to model the long-term trend component and a neural network to model the short-term fluctuations. By merging these, it captures both seasonal structure and nonlinear variations. The authors report that combined models outperformed even the deep nets alone, indicating that there is complementary value in classical methods for trend/seasonality alongside machine learning for residual patterns. They suggest that such hybrid approaches can leverage the strengths of each technique, ultimately yielding more accurate and reliable forecasts of energy demand peaks than either ARIMA or deep learning could achieve on their own. The success of the ARIMA-LSTM model underscores how modern deep learning can enhance (rather than replace) traditional time-series forecasting in the context of short-term energy predictions. Similar to this study \textcite{zhang2024} also produced individual ARIMA and LSTM models to forecast short-term power load. They also ended up combining them into a hybrid ARIMA-LSTM model to achieve the best model performance. This theme of hybrid models is a popular methodology in the energy forecasting literature and most of the research proves it superior to either model standalone \parencite{semmelmann2022, arslan2022, bashir2022, grandon2023}.

\subsection{Takeaways from the Literature}

The reviewed literature highlights several important patterns in short-term forecasting research. A wide range of modeling approaches have been applied to forecasting carbon emissions and related energy variables, from classical time-series models such as ARIMA and exponential smoothing to more advanced machine learning and deep learning techniques like LSTM and hybrid architectures. Traditional models remain valuable for their simplicity and interpretability, but often struggle with capturing the nonlinear relationships and high variability found in complex energy systems. Hybrid models that combine statistical and machine learning components show strong performance in various forecasting tasks, especially when they integrate feature selection, decomposition, or residual modeling.

While deep learning has been explored in related areas such as short-term load or wind forecasting, its application to carbon emissions forecasting (especially at a regional and hourly resolution) remains limited. Moreover, few studies focus specifically on system-level forecasting in highly renewable zones like DK1. This leaves an opportunity to explore how advanced models like LSTM perform in a context characterized by high wind penetration, weather-dependence, and dynamic electricity flows. These insights from prior research help frame the motivation and direction of the present thesis.

\subsection{Research Gap and Thesis Aim}

Despite the growing body of research on short-term forecasting of carbon emissions and related energy variables, a notable gap persists in region-specific applications, particularly in highly renewable-integrated power systems like Denmark's DK1 bidding zone. Most existing studies focus on larger national grids such as Germany, the UK, or pan-European systems \parencite{ostermann2024, bokde2021}, or they target demand-side forecasting tasks like short-term load or peak demand prediction \parencite{pierre2023, zhang2024}. While some research has been conducted for Denmark, such as the work by \textcite{leerbeck2020} in DK2, there is a lack of targeted modeling efforts for DK1, which is distinct in both its physical generation profile and operational constraints.

DK1 stands out due to its exceptionally high share of wind power, accounting for over 50\% of its electricity generation \parencite{wang2017,iea2023}. This introduces high temporal variability in the carbon emissions of its electricity mix. Unlike hydropower-dominated systems that can buffer such fluctuations through reservoir storage, DK1 relies on a combination of market mechanisms, cross-border interconnections, and dispatchable fossil generation to balance supply and demand \parencite{energistyrelsen}. These characteristics make carbon emissions forecasting particularly challenging in this zone, and suggest that traditional models, while still valuable, may not fully capture the nonlinear dynamics associated with variable renewables.

Furthermore, while deep learning and hybrid models, such as LSTM and ARIMA-LSTM, have been successfully applied to short-term energy forecasting, their use for hourly-resolution, system-level carbon emissions forecasting remains comparatively limited. In particular, regional studies that apply such methods to weather-sensitive, high-renewable zones like DK1 are largely missing from the literature.  Moreover, many existing studies do not incorporate localized operational characteristics such as cross-border flows, wind generation profiles or similar exogenous variables that may be critical for accurate carbon forecasting in complex systems like DK1.

This thesis addresses the identified research gap by developing a long short-term memory neural network model to forecast hourly, 24-steps-ahead system-level carbon emissions in Denmark's DK1 bidding zone. The model will be evaluated against both a naive persistence baseline and a traditional statistical approach such as ARIMA. By focusing on DK1, the study contributes a novel case to the literature, one that captures the unique forecasting challenges in a power system dominated by variable wind generation and reliant on market-based balancing and cross-border flows. LSTM is explored as a modern and complex method capable of modeling nonlinear and long-range temporal dependencies, making it a fitting choice for this setting. The aim is to assess whether such an advanced model can significantly reduce forecasting errors and to understand whether the complexity of LSTM is justified when compared to simpler, well-established alternatives.
