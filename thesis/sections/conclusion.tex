\thispagestyle{plain}
\section{Conclusion}

This thesis demonstrates how long short-term memory neural networks can be effectively designed to forecast short-term carbon emissions in Denmark's wind-dominated DK1 bidding zone, achieving statistically significant improvements over both simple benchmark models and traditional autoregressive approaches. Developing LSTM architectures tailored for wind-dominated grids address a critical research gap and provides practical forecasting solutions for grid operators and energy-intensive industries.

The optimal LSTM configurations identified through systematic hyperparameter optimization reveal consistent architectural patterns across forecasting horizons. Both the 1-hour and 24-hour models converged to 32 LSTM units processing 48-hour input sequences with identical learning rates of 0.0001, though requiring different architectural approaches. The 1-hour model employed a single-layer configuration for scalar predictions, while the 24-hour model utilized an encoder-decoder architecture for sequence-to-sequence forecasting. This architectural distinction reflects the fundamental difference in prediction complexity, where extended forecasting horizons benefit from dedicated components for feature extraction and sequence generation, enabling effective integration of both historical patterns and future exogenous variable forecasts available at prediction time.

Feature importance analysis validates theoretical merit-order principles while revealing a surprising insight: despite wind power providing over 50\% of DK1's electricity generation, forecasted solar photovoltaic production emerged as the most influential exogenous predictor. This highlights that predictable renewable sources can be more valuable for forecasting than larger but more variable ones. Solar generation follows reliable daily cycles with zero overnight production and predictable midday peaks, providing the model with systematic patterns that help anticipate when the grid will rely more heavily on carbon-intensive backup generation.

The LSTM models achieve substantial performance improvements across both forecasting horizons. For 24-hour ahead prediction, the encoder-decoder LSTM reduces RMSE by 11.3\% compared to optimally-configured ARIMA models and 25.4\% compared to naive persistence, with parallel improvements in MAE. The 1-hour model delivers more modest but statistically significant gains of 4.5\% over ARIMA. Diebold-Mariano tests confirm that these improvements are statistically significant rather than due to random variation.

The practical implications extend beyond academic interest to tangible economic and environmental benefits. Conservative estimates suggest that a medium-sized manufacturing company could achieve annual carbon reductions of 855 to 1,905 tonnes \cotwoe{} through optimized production scheduling enabled by accurate emission forecasts, representing cost savings of \euro72,700 to \euro161,900 at current carbon prices. For transmission system operators like Energinet, enhanced forecasting enables proactive grid management strategies that could reduce emergency reserve activations by 17 to 38 GWh annually, yielding operational savings of \euro3.2 to \euro7.1 million while avoiding 8,500 to 19,000 tonnes \cotwoe{} in emissions. These quantified benefits demonstrate that improved forecasting accuracy translates directly into both economic efficiency and climate impact across industrial and grid operations.

This thesis establishes LSTM neural networks as a viable machine learning technique for carbon emission forecasting in renewable-dominated power systems. As electricity grids worldwide transition toward higher renewable penetration levels similar to DK1's profile, the forecasting approaches developed offer transferable solutions for managing volatile clean energy systems while supporting both economic and climate objectives.
